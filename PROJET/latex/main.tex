% Rapport ESIEE — Reconnaissance d'objets en vidéos
\documentclass[12pt,a4paper]{article}

% Encodage et langue
\usepackage[utf8]{inputenc}
\usepackage[T1]{fontenc}
\usepackage[french]{babel}

% Mise en page et microtypographie
\usepackage{geometry}
\geometry{margin=2.5cm}
\usepackage{microtype}
\usepackage{hyperref}
\hypersetup{colorlinks=true, linkcolor=blue, urlcolor=blue, citecolor=blue}

% Outils utiles
\usepackage{graphicx}
\usepackage{booktabs}
\usepackage{csquotes}
\usepackage{enumitem}

% Style de paragraphe
\setlength{\parskip}{0.6em}
\setlength{\parindent}{0pt}

% Titre
\title{Reconnaissance d'objets en vidéos: \textit{État de l'art} et solutions Machine Learning}
\author{Entreprise: \emph{Nom de l'entreprise (fictive)}\\ Groupe: \emph{Membres du groupe}}
\date{ESIEE Paris — \today}

\begin{document}
\maketitle
\tableofcontents
\clearpage

% Première partie (Résumé exécutif + Introduction)
\input{sections/part1}

% Les autres parties seront ajoutées étape par étape
% Deuxième partie — État de l’art des méthodes

\section{État de l’art des méthodes}

\subsection{Famille YOLO (You Only Look Once)}
\paragraph{Principe} YOLO reformule la détection d’objets en une régression unique effectuée en une seule passe. L’image est divisée en grille, chaque cellule prédit directement boîtes englobantes et classes, assurant une inférence ultra-rapide adaptée au temps réel.

\paragraph{Évolution} \begin{itemize}[left=0pt]
  \item YOLOv1–v3: unification une étape, ancres, multi-échelles, backbone Darknet-53.
  \item YOLOv4: CSPNet, Mosaic/CutMix, entraînement SAT; 43.5\% mAP COCO.
  \item YOLOv5: implémentation PyTorch, auto-hyperparamètres, export ONNX/TensorRT.
  \item YOLOv8: tête \emph{anchor-free}, tâches unifiées, Distribution Focal Loss.
  \item YOLOv10: entraînement sans NMS via assignations doubles cohérentes; latence réduite.
  \item YOLOv11: blocs C3k2, SPPF; meilleures performances à paramètres égaux, tâches étendues.
\end{itemize}

\paragraph{Performances} En 2025, YOLOv11 couvre 39–55\% mAP avec 1.5–11 ms par image sur GPU T4 ($\approx$ 88–200+ FPS), offrant un excellent compromis précision/vitesse sur vidéo temps réel.

\paragraph{Avantages} \begin{itemize}[left=0pt]
  \item Vitesse exceptionnelle et latence faible.
  \item Précision compétitive face à des méthodes plus complexes.
  \item Moins de faux positifs grâce au contexte global.
  \item Polyvalence: détection, segmentation, classification, pose.
  \item Déploiement facilité (CPU/GPU/edge; export vers ONNX/TensorRT/CoreML).
\end{itemize}

\paragraph{Limites} \begin{itemize}[left=0pt]
  \item Petits objets/occlusions restent difficiles, surtout en variantes légères.
  \item Ressources GPU nécessaires pour les versions précises (l/x).
  \item Sensible aux conditions extrêmes (éclairage, flou).
  \item Inévitable \emph{trade-off} précision/vitesse selon la taille du modèle.
\end{itemize}

\paragraph{Applications typiques} \begin{itemize}[left=0pt]
  \item Surveillance temps réel, conduite autonome, retail analytics, inspection industrielle.
\end{itemize}

\paragraph{Type d’apprentissage} Supervision classique avec boîtes et labels; pertes localisation/classe/confiance. Entraînement moderne: augmentation (Mosaic/MixUp), multi-échelle, assignation dynamique.

\subsection{Architectures Transformer pour vidéo}
\paragraph{Contexte ViT→vidéo} ViT traite des patches comme tokens; en vidéo, il faut modéliser simultanément dimensions spatiales et temporelles pour suivre objets et actions sur plusieurs frames.

\paragraph{TGBFormer} Fusion Transformer + GraphFormer: dépendances globales et relations locales spatio-temporelles; \textbf{86.5\% mAP à 41 FPS} (ImageNet VID), viable pour haute précision avec contraintes temps réel modérées.

\paragraph{VideoGLaMM} Alignement multimodal pixel-niveau: relie descriptions textuelles et localisations précises dans la vidéo via encodeurs visuels duaux et décodeur spatio-temporel; ouvre recherche/annotation par langue naturelle.

\paragraph{ViViT et variantes} \begin{itemize}[left=0pt]
  \item Attention jointe spatio-temporelle (performante mais coûteuse).
  \item Encodeur factorisé (spatial puis temporel), plus efficace.
  \item Réduction temporelle/pooling central; AMViT (mémoire adaptative) pour longues vidéos.
\end{itemize}

\paragraph{Avantages} \begin{itemize}[left=0pt]
  \item Modélisation temporelle supérieure et compréhension contextuelle riche.
  \item Robustesse au mouvement rapide via agrégation multi-frame.
  \item Flexibilité multimodale (texte/audio).
\end{itemize}

\paragraph{Limites} \begin{itemize}[left=0pt]
  \item Complexité quadratique, mémoire KV coûteuse.
  \item Besoins élevés en données, latence d’inférence.
\end{itemize}

\paragraph{Type d’apprentissage} Principalement supervisé (pré-entraînement images/vidéos, fine-tuning). Forte émergence d’auto-supervision (prédiction futures, contrastif, réordonnancement).

\subsection{DETR et variants}
\paragraph{DETR original} \emph{Set prediction} end-to-end: backbone CNN + encodage positionnel → Transformer; \emph{object queries} interrogent les features, têtes prédisent classe/boîte; appariement bipartite (Hungarian) élimine le NMS. Convergence initiale lente.

\paragraph{MI-DETR} Remplace la cascade par interrogations parallèles (\emph{multi-time inquiries}) via têtes SA/CA/FFN indépendantes dont les sorties sont concaténées et projetées. Améliore extraction d’information et robustesse aux occlusions/variations.

\subsection{Agrégation temporelle inter-frames}
\paragraph{YOLO + agrégation de features} Sélection/agrégation multi-échelle de features sur plusieurs frames: \textbf{92.9\% AP50} à \textbf{30+ FPS}. Idéal quand la précision prime avec latence légère.

\paragraph{Exploitation inter-frames} \begin{itemize}[left=0pt]
  \item Réduction du bruit et des faux positifs isolés.
  \item Complétion d’occlusions temporaires et stabilité des identités.
  \item Bounding boxes et classes plus précises; suivi implicite.
\end{itemize}

\paragraph{Avantages} Performance sur mouvements rapides; robustesse aux conditions variables; tracking simplifié.

\paragraph{Limites} Coût mémoire (cache features), sensibilité aux changements brusques d’apparence, légère latence (2–5 frames).

\subsection{Approches émergentes}
\paragraph{SNNs (MSD)} Détection bio-inspirée économe en énergie: \textbf{62.0\% mAP} avec 7.8M paramètres et 6.43 mJ; Spiking-YOLO approche 98\% de Tiny YOLO avec consommation \textasciitilde280× inférieure.

\paragraph{Faible supervision} DOtA (détection 3D multi-agents sans annotations), PointSR (supervision point-level), DViN (vision-langage référentiel) réduisent massivement le coût d’annotation.

\paragraph{Potentiel futur} Déploiement embarqué/edge (Loihi/TrueNorth), réduction des coûts d’annotation (\textasciitilde50–90\%), meilleure scalabilité vers nouveaux domaines.

\subsection{Tableau comparatif synthétique}
\begin{table}[h]
  \centering
  \small
  \begin{tabular}{p{3cm} p{2.5cm} p{4cm} p{4cm} p{3.2cm}}
    \toprule
    \textbf{Méthode} & \textbf{mAP/FPS} & \textbf{Points forts} & \textbf{Limites} & \textbf{Cas d’usage} \\
    \midrule
    YOLO (v11) & 39–55\% / 88–200+ & Vitesse, simplicité, déploiement edge & Petits objets, GPU requis pour haute précision & Surveillance, embarqué, retail \\
    Transformers vidéo & 80–87\% / 30–40 & Contexte global, temporalité, multimodal & Complexité, mémoire, latence & Analyse avancée, annotation, recherche vidéo \\
    DETR / MI-DETR & 43–55\% / 20–40 & End-to-end, pas de NMS, robustesse & Convergence/latence, coût inférence & Scènes complexes, occlusions \\
    YOLO + agrégation & AP50 92.9 / 30+ & Précision stabilisée multi-frame & Mémoire, latence légère & Haute précision vidéo temps réel \\
    SNNs (MSD) & 62\% / très efficace & Énergie ultra-faible, puces neuromorphiques & Perf. inférieure, outillage limité & Embarqué batterie/IoT \\
    Faible supervision & — / — & Moins d’annotations (50–90\% gain) & Perf. dépend des signaux faibles & Nouvelles classes/domaines, prototypage \\
    \bottomrule
  \end{tabular}
\end{table}
% Troisième partie — Datasets disponibles

\section{Datasets disponibles}

\subsection{ImageNet VID}
\paragraph{Description} Benchmark standard pour la détection d'objets en vidéo, dérivé d'ImageNet et étendu aux séquences avec annotations temporelles cohérentes.

\paragraph{Caractéristiques} \begin{itemize}[left=0pt]
  \item \textasciitilde4000 séquences vidéo en train/val, 30 catégories d'objets (animaux, véhicules, objets du quotidien).
  \item Annotations en bounding boxes et labels de classe sur frames sélectionnées.
  \item Scènes naturelles: mouvements de caméra, changements d'échelle, occlusions, variations d'éclairage.
\end{itemize}

\paragraph{Forces} Standard largement adopté; annotations de haute qualité; défis vidéo réalistes; évalue la cohérence temporelle (ex.: TGBFormer 86.5\% mAP à 41 FPS; ClipVID 84.7\% mAP à 39.3 FPS).

\paragraph{Biais identifiés} Échelle limitée; surreprésentation de certaines catégories; séquences courtes; résolutions variables.

\paragraph{Utilisation recommandée} Benchmarking et validation par rapport à l'état de l'art avant application sur données spécifiques au domaine.

\subsection{YouTube-VOS}
\paragraph{Description} Dataset à large échelle pour segmentation vidéo (spatio-temporel), clips YouTube diversifiés.

\paragraph{Caractéristiques} \begin{itemize}[left=0pt]
  \item 4,453 clips (2018), 78 catégories; version 2021: 3,859 vidéos.
  \item Annotations pixel-level tous les 5 frames (≈6 FPS), objets multiples par clip (jusqu'à 5).
  \item 133,886 annotations (2018) ; 232k annotations (2021) sur 8,171 instances uniques.
\end{itemize}

\paragraph{Forces} Très grande échelle; diversité réaliste; 26 catégories de validation non vues (mesure de généralisation); masques précis.

\paragraph{Limites} Clips courts (3–6 s); annotations \emph{skip-frame}; complexité très variable; biais de popularité YouTube.

\paragraph{Utilisation recommandée} Entraînement générique de détection/segmentation vidéo; pré-entraînement avant fine-tuning; tâches pixel-level (édition, AR).

\subsection{COCO (Common Objects in Context)}
\paragraph{Description} Dataset d'images statiques massivement utilisé pour pré-entraîner les détecteurs avant adaptation vidéo.

\paragraph{Caractéristiques} \begin{itemize}[left=0pt]
  \item 330k images (200k annotées), 80 catégories ; \textasciitilde1.5M instances (≈47 objets/image).
  \item Annotations multi-tâches: bounding boxes, masques instance, keypoints (250k+ personnes), segmentation \emph{stuff}, 5 captions par image.
  \item Ensembles standardisés: Train2017 (118k), Val2017 (5k), Test2017 (20k).
\end{itemize}

\paragraph{Usage pour la vidéo} Pré-entraînement des backbones (ResNet, EfficientNet, Transformers); performances COCO corrèlent les capacités vidéo (ex.: YOLOv11x 54.7\% mAP, MI-DETR 52.4\% mAP).

\paragraph{Forces} Très grande échelle; intégration facilitée (PyTorch/TensorFlow/Ultralytics); métriques rigoureuses; scènes en contexte naturel.

\paragraph{Limites} Pas de temporalité; sous-représentation des très petits objets (<32×32) ; distribution de catégories déséquilibrée; annotations statiques.

\paragraph{Utilisation recommandée} Pré-entraînement incontournable pour tout détecteur ; fine-tuning sur petit dataset vidéo cible.

\subsection{DAVIS (Densely Annotated VIdeo Segmentation)}
\paragraph{Description} Benchmark haute qualité pour segmentation vidéo.

\paragraph{Caractéristiques} \begin{itemize}[left=0pt]
  \item DAVIS 2016: 50 séquences Full HD (1080p, 24 FPS) ; 2017: 90 séquences, multi-objets.
  \item Annotations pixel-level exhaustives sur tous les frames; attributs de défis annotés.
  \item Métriques: similitude région (J), précision contours (F), cohérence temporelle (T).
\end{itemize}

\paragraph{Forces} Masques \emph{pixel-perfect}; qualité vidéo 1080p ; benchmark standard reconnu; couverture systématique des défis.

\paragraph{Limites} Échelle très limitée; performances en voie de saturation; coût d'annotation prohibitif; focalisation segmentation binaire.

\paragraph{Utilisation recommandée} Benchmark/validation pour segmentation très précise; compléter par données domaine-spécifiques pour entraînement.

\subsection{OD-VIRAT}
\paragraph{Description} Benchmark large échelle pour détection en surveillance réaliste ; variantes Large (8.7M instances / 599,996 images) et Tiny (288,901 instances / 19,860 images).

\paragraph{Caractéristiques} \begin{itemize}[left=0pt]
  \item 10 scènes de surveillance (chantiers, parkings, rues), caméras statiques en hauteur.
  \item Objets de petite échelle; 5 catégories: Bike/Bicycle, Car, Carrying\_object, Person, Vehicle.
  \item Arrière-plans complexes; sampling 0-frame-skip (Large) vs. 30-frame (Tiny).
\end{itemize}

\paragraph{Forces} Conditions de surveillance authentiques; échelle massive (Large); benchmarking spécialisé; résolutions HD (1280×720, 1920×1080 à 25–30 FPS).

\paragraph{Limites} Domaine spécialisé (généralisation limitée) ; objets très petits difficiles ; seulement 5 catégories ; biais géographiques/temporalité.

\paragraph{Utilisation recommandée} Entraîner/évaluer des détecteurs pour surveillance ; utiliser Tiny pour prototypage rapide et Large pour entraînement robuste; adapter aux cas non-surveillance via fine-tuning.

\subsection{Tableau comparatif datasets}
\begin{table}[h]
  \centering
  \small
  \begin{tabular}{p{3cm} p{2.5cm} p{3.8cm} p{3.6cm} p{3.8cm} p{3.6cm}}
    \toprule
    \textbf{Dataset} & \textbf{Taille} & \textbf{Annotations} & \textbf{Domaine} & \textbf{Biais principaux} & \textbf{Usage recommandé} \\
    \midrule
    ImageNet VID & \textasciitilde4k vidéos / 30 classes & BBoxes par frame & Détection vidéo générique & Échelle limitée; surreprésentation classes; séquences courtes & Benchmarking détection vidéo \\
    YouTube-VOS & 4,453 clips / 78 cat. & Masques pixel-level (\emph{skip-frame}) & Segmentation vidéo générale & Clips courts; popularité YouTube; variabilité forte & Entraînement segmentation; pré-entraînement vidéo \\
    COCO & 330k images / 80 cat. & BBoxes; masques; keypoints; captions & Images statiques génériques & Pas de temporalité; petits objets rares; classes déséquilibrées & Pré-entraînement backbones; transfert vidéo \\
    DAVIS & 50–90 séquences (FHD) & Masques sur \textbf{tous} les frames & Segmentation vidéo précise & Échelle limitée; saturation performances; coût annotation & Validation précision segmentation \\
    OD-VIRAT & 8.7M inst. (Large) & BBoxes surveillance (5 cat.) & Surveillance réaliste & Petits objets; domaine étroit; biais géographique & Détecteurs surveillance; Tiny pour prototypage \\
    \bottomrule
  \end{tabular}
\end{table}
% Quatrième partie — Métriques de performance

\section{Métriques de performance}

L'évaluation rigoureuse des détecteurs d'objets vidéo repose sur des métriques complémentaires couvrant la localisation spatiale, la classification, et la vitesse (temps réel). Cette section synthétise les métriques standard, leurs interprétations et leur pertinence selon les cas d'usage.

\subsection{Métriques de précision spatiale}
\paragraph{IoU (Intersection over Union)} Mesure le chevauchement entre la box prédite et la vérité terrain. \emph{Définition} :
\[
\mathrm{IoU} = \frac{\mathrm{area}(B_{\text{pred}} \cap B_{\text{gt}})}{\mathrm{area}(B_{\text{pred}} \cup B_{\text{gt}})}
\]
Pour la classification binaire (présence/absence sur pixel ou instance), on rencontre l'approximation suivante :
\[\mathrm{IoU} = \frac{TP}{TP + FP + FN}\]
\emph{Interprétation} : \begin{itemize}[left=0pt]
  \item IoU = 0 : aucun chevauchement
  \item 0 < IoU < 0.5 : localisation imprécise
  \item IoU ≥ 0.5 : détection généralement considérée valide
  \item IoU ≥ 0.75 : haute précision de localisation
  \item IoU = 1 : correspondance parfaite
\end{itemize}
\emph{Pertinence} : métrique fondamentale de localisation, pénalise sous- et sur-détection. \emph{Limites} : forte sensibilité aux petits objets (< 32×32), n'encode pas l'exactitude de la classe.

\paragraph{Précision et Rappel} \emph{Définitions} :
\[\mathrm{Precision} = \frac{TP}{TP + FP} \quad;\quad \mathrm{Recall} = \frac{TP}{TP + FN}\]
\emph{Trade-off} selon le seuil de confiance : \begin{itemize}[left=0pt]
  \item Seuil élevé : précision ↑, rappel ↓ (modèle conservateur)
  \item Seuil bas : précision ↓, rappel ↑ (modèle permissif)
\end{itemize}
\emph{Choix contextuel} : \begin{itemize}[left=0pt]
  \item Surveillance : privilégier rappel élevé (éviter FN)
  \item Retail analytics : équilibre précision/rappel (F1)
  \item Conduite autonome : précision \textbf{et} rappel très élevés (> 0.98)
\end{itemize}

\subsection{Métriques agrégées}
\paragraph{F1-Score} Moyenne harmonique précision–rappel :
\[\mathrm{F1} = \frac{2\,\mathrm{Precision}\cdot\mathrm{Recall}}{\mathrm{Precision}+\mathrm{Recall}} = \frac{2TP}{2TP + FP + FN}\]
Pénalise les modèles déséquilibrés, utile sur datasets avec classes majoritaires/minoritaires.

\paragraph{Average Precision (AP) et Mean Average Precision (mAP)} \emph{AP} résume la courbe précision–rappel (aire sous courbe). \emph{mAP} : moyenne des AP sur toutes les classes.
\begin{itemize}[left=0pt]
  \item \emph{mAP@0.50} (VOC) : seuil IoU unique à 0.50, indulgent sur la localisation.
  \item \emph{mAP@0.50:0.95} (COCO) : moyenne sur IoU ∈ \{0.50,...,0.95\}, plus exigeant (souvent ~15–20 pts \emph{en dessous} de mAP@0.50 pour un même modèle).
\end{itemize}
\emph{Pertinence} : standard de comparaison inter-modèles (COCO, ImageNet VID). \emph{Limites} : masque les faiblesses sur classes rares, biais si objets « faciles » dominent, peu sensible à la consistance temporelle.

\subsection{Métriques de vitesse}
\paragraph{FPS (Frames Per Second)} Nombre d'images traitées par seconde — critère clé pour le temps réel.
\begin{itemize}[left=0pt]
  \item < 15 FPS : trop lent (offline ou non critique)
  \item 15–25 FPS : acceptable si tolérance aux retards
  \item 25–30 FPS : minimum pour fluidité perceptuelle
  \item 30–60 FPS : idéal pour temps réel (surveillance, robotique)
  \item > 60 FPS : excellent, marge pour tracking/post-traitements
\end{itemize}
Exemple : TGBFormer atteint 86.5\% mAP à 41 FPS (bon équilibre précision/vitesse).

\paragraph{Latence} Délai capture→détection (prétraitement + inférence + post-traitement).
\[\mathrm{Latence} = t_{\text{détection disponible}} - t_{\text{capture}}\]
\emph{Relation} : \textit{FPS} mesure le débit, \textit{latence} la réactivité (pipeline : FPS élevé possible avec latence élevée).
\emph{Contraintes typiques} : \begin{itemize}[left=0pt]
  \item Conduite autonome : < 50 ms \quad Drone : < 30 ms
  \item Surveillance : < 200 ms \quad Analytics offline : non critique
\end{itemize}
\emph{Variabilité matériel} : GPU A100/V100 ≫ RTX 3060 ≫ CPU ≫ Edge (Jetson/mobile). Optimisations courantes : quantification (FP32→INT8), pruning, distillation, TensorRT/ONNX (gain 2–5×; perte mAP ~1–3 pts).

\subsection{Choix des métriques selon cas d'usage}
Le choix doit refléter coûts relatifs des erreurs et contraintes opérationnelles.
\begin{itemize}[left=0pt]
  \item \textbf{Surveillance sécurité} : rappel élevé, mAP@0.50, FPS ≥ 25.
  \item \textbf{Conduite autonome} : mAP@0.50:0.95, FPS ≥ 30, latence < 50 ms.
  \item \textbf{Retail analytics} : précision, rappel, F1 élevé (≤ temps réel strict).
  \item \textbf{Inspection industrielle} : précision élevée, IoU/mAP@0.75 (localisation fine).
  \item \textbf{Analyse sportive} : mAP@0.50, FPS ≥ 30, F1 (fluidité prioritaire).
  \item \textbf{Santé/Médical} : précision très élevée, IoU, mAP@0.75 (offline acceptable).
\end{itemize}

\subsection{Tableau récapitulatif : Métriques par cas d'usage}
\begin{table}[h]
  \centering
  \small
  \begin{tabular}{p{3.2cm} p{4.0cm} p{3.2cm} p{5.6cm}}
    \toprule
    \textbf{Cas d'usage} & \textbf{Métriques prioritaires} & \textbf{Seuils typiques} & \textbf{Justification} \\
    \midrule
    Surveillance & Rappel, mAP@0.50, FPS & Rappel > 0.95, FPS ≥ 25 & Éviter FN critiques, localisation modérée suffisante, temps réel requis. \\
    Conduite autonome & mAP@0.50:0.95, Latence & Latence < 50 ms, FPS ≥ 30 & Précision \& réactivité maximales, erreurs très coûteuses. \\
    Retail analytics & Précision, Rappel, F1 & F1 > 0.85 & Statistiques fiables, temps réel strict non critique (15–20 FPS). \\
    Inspection industrielle & IoU, mAP@0.75, Précision & IoU/mAP élevés & Localisation fine pour action robotique, coûts FP/FN élevés. \\
    Analyse sportive & mAP@0.50, FPS, F1 & FPS ≥ 30 & Fluidité visuelle, corrections possibles en post-prod. \\
    Santé/Médical & Précision, IoU, mAP@0.75 & Précision → max & Diagnostic fiable, minimiser FP/FN, offline acceptable. \\
    \bottomrule
  \end{tabular}
\end{table}
\section{Types d'apprentissage}\label{sec:types-apprentissage}

Cette partie présente les paradigmes d'apprentissage pertinents pour la détection d'objets en vidéo, leurs coûts et leurs compromis de performance.

\subsection{Apprentissage supervisé}
\textbf{Principe.} Données annotées (bounding boxes + classes) et optimisation d'une perte de localisation \& classification.

\textbf{Stratégies d'annotation.} Annotation \emph{dense} (toutes les frames) vs \emph{skip-frame} (p.ex. 1~FPS sur une vidéo 30~FPS, coût réduit \textasciitilde97\%). Les frames intermédiaires sont apprises par propagation temporelle.

\textbf{Méthodes concernées.} YOLO (v1--v11), Transformers vidéo (p.ex. ViViT), DETR/MI-DETR ; pré-entraînement sur COCO/ImageNet puis fine-tuning domaine.

\textbf{Avantages.} Précision maximale sur benchmarks ; performance prédictible si données suffisantes ; transfert efficace ; outillage mature (PyTorch, CVAT, Labelbox).

\textbf{Limites.} Coût d'annotation élevé ; biais humains ; généralisation limitée hors distribution ; dépendance au domaine (changement de classes/contexte $\Rightarrow$ nouvelles annotations).

\textbf{Coût indicatif.} 1000 vidéos, 1~min, skip-frame 1~FPS $\Rightarrow$ \textasciitilde60k frames. Scénario modéré (4 objets/frame) : coût total \textasciitilde\$23.6k (plateforme + main d'œuvre). Segmentation pixel-level \textasciitilde$3\times$ plus coûteuse.

\subsection{Apprentissage non supervisé}
\textbf{Principe.} Découverte de représentations à partir de signaux vidéo naturels (cohérence temporelle, continuité spatiale) sans labels.

\textbf{Exemples.} Cohérence par tracking ; segmentation auto-supervisée (p.ex. SOLV) ; random walks sur graphes ; apprentissage égocentrique.

\textbf{Avantages.} Zéro annotation ; exploitation de corpus massifs (web) ; découverte de patterns ; robustesse améliorée aux changements de domaine.

\textbf{Limites.} Précision inférieure au supervisé ; conception/entraînement plus complexes ; validation difficile sans ground truth ; sensibilité aux biais des données.

\subsection{Approches hybrides}
\textbf{Faible supervision.} Point-level (p.ex. PointSR), image-level (WSOD, MIL), pseudo-labels raffinés (p.ex. W2N), collaboration segmentation-détection (p.ex. SDCN). Réduction de coût de 80--95\% vs bounding boxes.

\textbf{Auto-supervision.} Tâches de pré-texte (reconstruction masquée), adaptation de scène par auto-enseignement, cohérence multi-vues (p.ex. DOtA). Foundation models (SAM, CLIP) facilitent zero/few-shot.

\textbf{Avantages.} Coûts drastiquement réduits ; performances \textasciitilde85--95\% du supervisé ; meilleure scalabilité.

\textbf{Limites.} Maturité industrielle inégale ; léger gap de performance (5--15\%) inacceptable en cas critique ; hyperparamètres supplémentaires.

\subsection{Recommandation stratégique}
\textbf{Court terme (0--6~mois).} Supervisé pour déploiement rapide : modèles pré-entraînés (YOLOv11, MI-DETR), fine-tuning sur 500--2000 vidéos, annotation skip-frame 1~FPS ; privilégier qualité d'annotation et pré-annotation (SAM) pour \textasciitilde30--50\% de gain.

\textbf{Moyen terme (6--18~mois).} Explorer faible supervision sur un sous-ensemble ; comparer vs supervisé ; si \(\ge 90\%\) de la performance, migrer progressivement (hybride : 10--20\% supervisé complet + 80--90\% faible).

\textbf{Long terme (18+~mois).} Auto-supervision \& foundation models ; évaluer régulièrement zero/few-shot et arbitrer coût/performance.

\begin{table}[h]
  \centering
  \small
  \begin{tabular}{l l l l}
    \toprule
    Application & Perf. min. & Supervision & Justification \\
    \midrule
    Conduite autonome & $>$98\% P/R & Supervisé complet & Criticité sécurité \\
    Surveillance & $>$95\% R & Supervisé + Faible & Événements critiques \\
    Retail & $>$85\% F1 & Faiblement supervisé & Scalabilité prioritaire \\
    Médical & $>$97\% Précision & Supervisé complet & Réglementation stricte \\
    Sport & $>$80\% mAP & Faible/Auto & Volume élevé \\
    Inspection qualité & $>$90\% Précision & Hybride & Selon sévérité défauts \\
    \bottomrule
  \end{tabular}
  \caption{Matrice décisionnelle pour le choix du paradigme.}
\end{table}

\noindent\textbf{ROI simplifié.}
\[
  \mathrm{ROI} = \frac{(V \times P) - C}{C},
\]
\noindent où \(V\) est la valeur business, \(P\) la performance (normalisée), \(C\) le coût d'annotation. Une baisse de \(C\) de 80\% pour \(\sim10\%\) de perte de \(P\) améliore fortement le ROI pour des applications non critiques.
\section{Vie priv\'ee et s\'ecurit\'e}\label{sec:privacy-security}

Cette partie pr\'esente le cadre RGPD et les mesures techniques pour un d\'eploiement \`a la fois efficace et conforme de la d\'etection d'objets en vid\'eo.

\subsection{Cadre r\'eglementaire RGPD}
\textbf{Surveillance vid\'eo = donn\'ees personnelles.} Les flux vid\'eo capturent des \'el\'ements identifiants (visages, silhouettes). Tout traitement (collecte, conservation, analyse ML) est soumis au RGPD.

\textbf{Bases l\'egales (Art.~6(1)).} Consentement (rarement applicable), Contrat (cas limit\'es), Obligation l\'egale (sp\'ecifique), Int\'er\^ets vitaux (urgence), Mission d'int\'er\^et public (autorites), Int\'er\^et l\'egitime (le plus courant, sous conditions de n\'ecessit\'e et proportionnalit\'e). 

\begin{table}[h]
  \centering
  \small
  \begin{tabular}{l l l}
    \toprule
    Base l\'egale & Applicabilit\'e & Remarques \\
    \midrule
    Consentement & Rare & Difficult\'e de refus libre \\
    Contrat & Limite & N\'ecessit\'e stricte \\
    Obligation l\'egale & Sp\'ecifique & Secteurs r\'eglement\'es \\
    Int\'er\^ets vitaux & Urgence & Temporaire \\
    Int\'er\^et public & Autorit\'es & Service public \\
    Int\'er\^et l\'egitime & Fr\'equent & N\'ecessit\'e + proportionnalit\'e + information \\
    \bottomrule
  \end{tabular}
  \caption{Synth\`ese des bases l\'egales applicables \`a la vid\'eosurveillance.}
\end{table}

\textbf{Droits des personnes.} Information (signal\'etique), acc\`es, rectification, effacement, opposition, limitation.

\subsection{\'Evaluation d'impact (DPIA)}
Obligatoire si risque \'elev\'e (Art.~35), typiquement pour une surveillance syst\'ematique \`a grande \'echelle en zone publique. Contenu: description du traitement, n\'ecessit\'e/proportionnalit\'e, analyse des risques (probabilit\'e \& gravite), mesures d'att\'enuation (techniques \& organisationnelles). Consultation de la CNIL si risque r\'esiduel \'elev\'e (Art.~36).

\subsection{Mesures techniques de protection}
\textbf{Privacy by Design.} Int\'egrer la protection d\`es la conception: edge/federated learning, anonymisation par d\'efaut, champs de vue limit\'es, s\'eparation pr\'ecoce des identit\'es.

\textbf{Anonymisation \& pseudonymisation.} Privil\'egier l'anonymisation irr\'eversible (floutage/pixellisation/masquage non r\'eversible). Entra\^iner sur vid\'eos anonymis\'ees; des travaux (p.~ex. E2PRIV) montrent des performances de d\'etection quasi identiques tout en r\'eduisant fortement les risques d'identification.

\textbf{Chiffrement.} En transit: TLS~1.2+, HTTPS/VPN. Au repos: AES-256, gestion des cl\'es s\'ecuris\'ee (rotation, isolement).

\textbf{Contr\^ole d'acc\`es.} Moindre privil\`ege, MFA, journaux d'audit complets et inviolables.

\textbf{Dur\'ees de conservation.} 72~h usuelles pour s\'ecurit\'e g\'en\'erale; jusqu'\`a 30~jours si justifi\'e. Suppression automatique \`a l'\'ech\'eance.

\subsection{Technologies alternatives}
\textbf{LiDAR.} Donn\'ees 3D sans texture ni couleur, d\'etection d'objets efficace avec exposition minimale des identit\'es. Co\^ut en baisse; utile pour p\'erim\`etres sensibles, comptage, intrusions.

\subsection{Donn\'ees sensibles \& minimisation}
Cat\'egories Art.~9 (biom\'etrie, sant\'e, opinions, etc.) \`a \'eviter. Si in\'evitable: satisfaire une base l\'egale (Art.~6) \emph{et} une exception (Art.~9(2)). Appliquer la minimisation (angles/cadrage, r\'esolution suffisante, d\'esactivation des fonctions non n\'ecessaires, masquage permanent des zones non pertinentes).

\subsection{Checklist de conformit\'e}
\begin{enumerate}[leftmargin=*]
  \item DPIA compl\'et\'ee, risque r\'esiduel acceptable document\'e.
  \item Base l\'egale identifi\'ee (souvent int\'er\^et l\'egitime) et test de proportionnalit\'e r\'ealis\'e.
  \item Privacy by Design impl\'ement\'e, chiffrement et contr\^ole d'acc\`es en place.
  \item Transparence: signal\'etique conforme, notice vie priv\'ee, proc\'edures d'exercice des droits.
  \item Dur\'ees de conservation configur\'ees (72~h \`a 30~j) avec suppression automatique.
  \item Gouvernance: DPO consult\'e, formation des \'equipes, contrats sous-traitants conformes, plan de r\'eponse aux incidents.
\end{enumerate}
% Partie 7 — Défis et limites
\section{Défis et limites}

\subsection{Défis techniques}
La reconnaissance d’objets en vidéo s’appuie sur des modèles confrontés à des scènes et des prises de vue très variées. Plusieurs défis techniques majeurs se dégagent :

\begin{itemize}
  \item \textbf{Objets petits ou fortement occultés.} Les objets de très petite taille (moins de 32\,\texttimes{}32 pixels) ou partiellement cachés posent un problème de résolution et de visibilité. Ils génèrent peu de pixels utiles pour l’extraction de caractères discriminants, réduisant la fiabilité de la détection, tandis que l’occultation partielle rend difficile la distinction de l’objet de l’arrière-plan.

  \item \textbf{Variations d’éclairage et conditions météo.} Les changements brusques d’éclairage (passage du soleil à l’ombre, contre-jour) et les conditions météorologiques extrêmes (pluie, brouillard, neige) altèrent la qualité d’image et dégradent les performances de modèles formés sur des images claires et stables. Ces variations requièrent des stratégies d’augmentation de données et des prétraitements dynamiques (normalisation adaptative).

  \item \textbf{Mouvement rapide et flou de mouvement.} Les objets se déplaçant rapidement, ou la caméra en mouvement, génèrent du flou cinétique qui dilue les contours et rend la localisation approximative. Les architectures à une seule passe (YOLO) sont particulièrement sensibles au flou, tandis que les transformers bénéficient de l’agrégation temporelle mais souffrent d’une latence accrue liée au traitement des séquences.

  \item \textbf{Arrière-plans complexes et scènes encombrées.} Dans des environnements urbains ou industriels chargés, les objets cibles peuvent se confondre avec des éléments de décor similaires (panneaux, machines, mobilier). Les modèles doivent distinguer les objets pertinents malgré de nombreux distracteurs et textures variées, ce qui nécessite des capacités de contextualisation globale avancées.

  \item \textbf{Compromis précision vs vitesse pour le temps réel.} Les applications de surveillance et de conduite autonome exigent à la fois une haute précision et une faible latence. Les CNN comme YOLO offrent une vitesse élevée (\(>88\,\text{FPS}\) pour YOLOv11) mais peuvent sacrifier de la précision sur petits objets ou scènes complexes, tandis que les transformers vidéo (TGBFormer, ViViT) améliorent la précision multi-cadres au prix d’un débit réduit (\(25\text{–}40\,\text{FPS}\)) et d’une forte consommation mémoire. Trouver l’équilibre adéquat reste un défi permanent.
\end{itemize}

\subsection{Biais des datasets}
Les datasets publics présentent des biais pouvant limiter la généralisabilité et l’équité des modèles :

\begin{itemize}
  \item \textbf{Déséquilibre de classes.} Un nombre disproportionné d’images pour certaines catégories (personne, véhicule, chien) entraîne des modèles surspécialisés au détriment d’objets moins fréquents. Par exemple, ImageNet VID et COCO surreprésentent certaines races de chiens et types de véhicules, biaisant les performances selon la classe.

  \item \textbf{Biais géographiques et culturels.} La majorité des vidéos provient de pays industrialisés (États-Unis, Europe, Asie de l’Est), exposant peu les modèles à des architectures, vêtements, véhicules ou scènes d’autres régions. Les systèmes entraînés sur ces données peuvent mal détecter des objets ou comportements spécifiques à des environnements différents.

  \item \textbf{Biais temporels.} Beaucoup de datasets datent de plus de cinq ans. Les objets et scènes évoluent rapidement (design de véhicules, styles vestimentaires, nouveaux dispositifs urbains). Les modèles risquent de manquer de sensibilité aux objets récents ou aux évolutions d’infrastructure, affectant leur pertinence en production.

  \item \textbf{Impact sur la généralisation et l’équité.} Ces biais mènent à une généralisation limitée et à des inégalités de performance selon les contextes et les populations filmées. Un détecteur peut fonctionner parfaitement sur des scènes diurnes occidentales mais échouer sur des environnements nocturnes ou ruraux.

  \item \textbf{Stratégies de mitigation.} \emph{(i)} \textit{Augmentation de données} : simuler conditions d’éclairage, flou, perspective et objets rares pour enrichir le spectre d’exemples. \emph{(ii)} \textit{Ré-échantillonnage} : équilibrer les classes en sur-échantillonnant les catégories sous-représentées ou en sous-échantillonnant les classes dominantes. \emph{(iii)} \textit{Datasets diversifiés} : combiner plusieurs sources (ImageNet VID, YouTube-VOS, OD-VIRAT, datasets locaux) couvrant variétés géographiques, culturelles et temporelles. \emph{(iv)} \textit{Validation out-of-distribution} : tester la robustesse sur des vidéos hors domaine d’entraînement pour mesurer la généralisation et détecter les zones de faiblesse.
\end{itemize}

\paragraph{Conclusion.} La prise en compte proactive de ces défis et biais est essentielle pour développer des solutions fiables, équitables et robustes dans des contextes réels variés.
% Partie 8 — Recommandations et faisabilité
\section{Recommandations et faisabilité}

\subsection{Matrice décisionnelle par cas d’usage}
\begin{table}[ht]
  \centering
  \small
  \begin{tabular}{p{3.5cm} p{4.2cm} p{4.1cm} p{4.4cm} p{4.2cm}}
    \toprule
    Cas d’usage & Architecture recommandée & Datasets & Métriques prioritaires & Conformité RGPD \\
    \midrule
    Surveillance temps réel & YOLOv11 (mode nano à média) & ImageNet VID, OD-VIRAT & Rappel élevé, mAP@0.50, FPS $\geq$ 25 & Intérêt légitime, DPIA obligatoire \\
    Analyse retail & TGBFormer & YouTube-VOS, COCO & Précision, mAP@0.50:0.95, F1-Score & Intérêt légitime, panneaux informatifs \\
    Conduite autonome & YOLO + agrégation de features & KITTI, BDD & mAP@0.50:0.95, FPS $\geq$ 30, Latence $<$ 50 ms & Obligation légale ou mission publique \\
    \bottomrule
  \end{tabular}
  \caption{Matrice décisionnelle par cas d’usage}
\end{table}

\subsection{Roadmap d’implémentation}
\begin{enumerate}[leftmargin=*]
  \item \textbf{Phase 1 (0–3 mois).} POC avec YOLOv11 sur ImageNet VID et OD-VIRAT en mode \emph{skip-frame}; évaluation des performances baseline (mAP, rappel, FPS); réalisation de la première DPIA pour cadrer les obligations RGPD.
  \item \textbf{Phase 2 (3–6 mois).} Collecte et annotation interne de 500–1\,000 vidéos spécifiques (\emph{skip-frame} 1 FPS); \emph{fine-tuning} du modèle sur données internes; intégration progressive de mesures Privacy by Design (anonymisation, chiffrement).
  \item \textbf{Phase 3 (6–12 mois).} Déploiement en production sur environnement restreint (edge ou cloud); monitoring continu (tableaux de bord mAP/FPS) et audit RGPD; ajustement des seuils et optimisation des pipelines d’inférence.
  \item \textbf{Phase 4 (12+ mois).} Optimisation des modèles (pruning, quantification) pour \emph{edge deployment}; exploration de l’apprentissage faiblement supervisé (PointSR) et auto-supervisé (DOtA); scalabilité vers de nouveaux sites et cas d’usage.
\end{enumerate}

\subsection{Estimation des ressources}
\paragraph{Humaines} 1–2 Data Engineers (collecte, annotation, pipelines données); 1 Machine Learning Engineer (fine-tuning, optimisation, déploiement); 1 expert conformité RGPD (DPIA, audits, procédures).

\paragraph{Techniques} GPU NVIDIA A100/V100 (ou cluster cloud équivalent) pour entraînement et inférence; stockage 5–10 TB pour datasets vidéo et temporaires; infrastructure cloud ou on-premise Kubernetes pour scalabilité.

\paragraph{Financières} Licences éventuelles pour datasets commerciaux (surveillance privée); coût annotation externe: $10–$25/h annotateur; dépenses d’infrastructure cloud: ~$5\,000–$10\,000/mois; formation et ateliers RGPD: $20\,000–$30\,000 initiaux.

\paragraph{Temporelles} 6–12 mois pour déploiement complet et stabilisation; itérations trimestrielles pour revue performance et conformité.

\subsection{Risques et mitigation}
\begin{description}[leftmargin=!,labelwidth=3.6cm]
  \item[\textbf{Risque technique}] Performances insuffisantes sur cas réels (petits objets, flou). \emph{Mitigation} : validation POC sur données internes, ajustement d’architecture (agrégation).
  \item[\textbf{Risque conformité}] Violations RGPD (surveillance illégale). \emph{Mitigation} : DPIA rigoureuse, consultation DPO, mise en place Privacy by Design.
  \item[\textbf{Risque budget}] Dépassement des coûts d’annotation et d’infrastructure. \emph{Mitigation} : phases incrémentales avec KPI clairs, basculement partiel vers supervision faible.
  \item[\textbf{Risque adoption}] Résistance des utilisateurs internes (sécurité, IT). \emph{Mitigation} : ateliers de sensibilisation, documentation des bénéfices, support technique dédié.
\end{description}

\subsection{Conclusion sur la faisabilité}
Les technologies de détection d’objets vidéo sont désormais matures et accessibles, adaptées à divers cas d’usage. La combinaison d’architectures éprouvées (YOLOv11, TGBFormer) et de jeux de données standard facilite une intégration rapide. La conformité RGPD est réalisable via une DPIA initiale, des mesures de Privacy by Design et un suivi continu. Un déploiement phasé permet de maîtriser coûts et risques, avec un ROI positif dès la phase POC. \textbf{Recommandation} : lancer un projet pilote supervisé, puis migrer progressivement vers des approches hybrides pour étendre le système à grande échelle.
% Partie 9 — Conclusion
\section{Conclusion}

La reconnaissance d’objets dans les flux vidéo repose désormais sur un panorama riche de solutions ML, allant des détecteurs unifiés à une passe (YOLO) aux architectures Transformer vidéo, en passant par des approches hybrides combinant supervision faible et auto-supervision. En 2025, YOLOv11 demeure une référence pour les applications nécessitant une vitesse extrême, tandis que les Transformers vidéo (TGBFormer, ViViT) offrent une précision et une compréhension temporelle supérieures au prix d’une plus grande complexité. Les méthodes émergentes, telles que les \emph{Spiking Neural Networks} et les stratégies faiblement supervisées (PointSR, DOtA), promettent de réduire significativement les coûts d’annotation et d’ouvrir de nouveaux cas d’usage embarqués.

Les jeux de données disponibles sont diversifiés et volumineux : ImageNet VID et YouTube-VOS couvrent des scènes variées avec annotations (boîtes et masques), COCO fournit une base d’images statiques dense, DAVIS sert de référence pour la segmentation au niveau pixel, et OD-VIRAT expose les défis de la surveillance réaliste. Les métriques standardisées (IoU, mAP@0.50:0.95, F1-Score, FPS, latence) assurent une comparaison rigoureuse entre modèles.

Le cadre RGPD impose un impératif de conformité qui reste réalisable : choisir une base légale adaptée (intérêt légitime), conduire une DPIA, appliquer des principes de \emph{Privacy by Design}, et mettre en œuvre chiffrement, anonymisation et procédures d’exercice des droits. La combinaison d’un déploiement phasé, d’une architecture technique robuste et d’un suivi juridique régulier assure la faisabilité du projet.

\paragraph{Perspectives à moyen et long terme} L’évolution du domaine s’oriente vers :
\begin{itemize}
  \item des approches hybrides optimisant conjointement coût et performance ;
  \item l’\emph{edge computing} pour réduire la latence et préserver la vie privée ;
  \item des techniques de traitement temps réel toujours plus efficaces.
\end{itemize}
Ces perspectives ouvrent la voie à des systèmes de reconnaissance d’objets vidéo performants, scalables et respectueux des droits fondamentaux.
% ...

\end{document}